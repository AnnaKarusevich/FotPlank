\input{text/diss}

\begin{document}

\def\labauthors{Войтович Д.А., Карусевич А.А., Разова А.А.}
\def\labgroup{430}
\def\labnumber{1}
\def\labtheme{Исследование фотоэффекта и измерение постоянной Планка}
\renewcommand{\vec}{\mathbf}
\renewcommand{\Re}{\operatorname{Re}}
\renewcommand{\Im}{\operatorname{Im}}
\renewcommand{\phi}{\varphi}
\renewcommand{\hat}{\widehat}

\input{text/titlepage}

\section{Теоретическая часть}
\subsection*{Фотоэффект}
В настоящее время в науке, технике и быту широкое применение получили фотоэлементы - электровакуумные или полупроводниковые приборы, преобразующие энергию электромагнитного излучения оптического диапазона в электрическую. Действие этих приборов основано на использовании фотоэлектрических эффектов (фотоэффектов). 
Существует два вида фотоэффекта: внешний и внутренний. Внешний фотоэффект (фотоэлектронная эмиссия) заключается в испускании поверхностью тела электронов во внешнее пространство (вакуум или газ) под действием падающей на эту поверхность световой энергии. Внутренним фотоэффектом называется изменение под действием световой энергии проводимости тел вследствие появления добавочных электронов проводимости, либо эффект разделения зарядов в облучаемом светом теле, что приводит к возникновению электродвижущей силы (фотогальванический эффект). Внешний фотоэффект используется в вакуумных и газонаполненных фотоэлементах, а также в фотоумножителях; внутренний - в полупроводниковых фотоэлементах: фото резисторах, фотодиодах и фототранзисторах (в том числе, в солнечных батареях).
В данной работе исследуется внешний фотоэффект. При этом используется вакуумный фотоэлемент, имеющий два электрода - катод, эмитирующий электроны под действием света, и анод - коллектор электронов. Если на анод подать положительный потенциал, то во внешней цепи прибора потечет ток, называемый фототоком. Величина фототока зависит от интенсивности света и материала катода. При увеличении потенциала анода фототок сначала увеличивается, а затем достигаеит насыщения (все электроны, испускаемые катодом, достигают анода, и при дальнейшем увеличении анодного напряжения величина тока не меняется).
Основные законы внешнего фотоэффекта, установленные экспериментально, состоят в следующем.
\begin{enumerate}
 	\item Величина фототока в режиме насыщения при неизменном спектральном составе излучения прямо пропорциональна интенсивности падающего света (закон Столетова).
	 \item Для каждого вещества существует длинноволновая граница фотоэффекта $\lambda_0$, за которой (при $\lambda>\lambda_0$) фотоэмиссия не наблюдается.
	 \item Максимальная кинетическая энергия электронов $W_{max}$ при фотоэффекте линейно возрастает с увеличением частоты падающего света $\nu$ и не зависит от его интенсивности: $W\propto \nu$.
\end{enumerate}

Эти законы просто объясняются на основании квантовой теории света и электронной теории твердого тела. Согласно квантовой теории, свет имеет дуальную природу: он обладает не только волновыми свойствами, но и свойствами частиц и может быть представлен в виде потока квантов света (фотонов). Энергия одного фотона пропорциональна частоте соответствующей волны: $E=h \nu$, где $h$-- постоянная Планка, по современным данным равная $6.6260755(40)\cdot10^{-34}$ Дж$\cdot$с

Излучение света на элементарном уровне идет отдельными квантами. Рассмотрим процесс фотоэмиссии из металла Поскольку каждый фотон действует на электроны твердого тела независимо от других фотонов, причем существует определенная вероятность $Р$ того, что это действие приведет к эмиссии электрона, то при попадании на катод $N$ фотонов в секунду электронный ток с него составит $n=NP$ электронов в секунду. Интенсивность падающего на фотокатод света согласно квантовой теории также пропорциональна $N: I\propto Nhv$. Таким образом образом объясняется закон Столетова.

\begin{figure}[h]
	\centering
	\includegraphics[]{fig/fig11}
	\caption{}
	\label{fig:1}
\end{figure}
Согласно электронной теории твердого тела, электроны проводимости в металле находятся в потенциальной яме и при достаточно низкой температуре равномерно распределены по энергиям, заполняя потенциальную яму до некоторого уровня (см. рис. \ref{fig:1}).
Наименьшая энергия, необходимая электрону для выхода в вакуум, называется работой выхода: $U_{\text{вых}} = e\phi$, где $е$ - заряд электрона. То же название обычно употребляют и для потенциала $\phi$, так как энергию в атомной физике принято измерять в электрон-вольтах. Максимальная энергия, которую может получить электрон при соударении с фотоном, равна $h \nu$. Очевидно, при $h \nu<e\phi$ фотоэмиссия невозможна, что объясняет существование красной границы.

Из рис.\ref{fig:1} также следует, что максимальная кинетическая энергия электрона равна
\begin{equation}
	\label{eq:1}
 	W_{max}=h \nu -e\phi
 \end{equation} 

Это уравнение (уравнение Эйнштейна) объясняет третий закон
фотоэффекта. Электроны, выбиваемые с более глубоких энергетических уровней, имеют меньшую кинетическую энергию.

Максимальную кинетическую энергию электронов $W_{max}K$ можно определить, если между анодом и катодом фотоэлемента создать тормозящее электроны поле. Для этого на анод подается отрицательный по отношению к катоду потенциал $V$. Вылетевшие из фотокатода электроны имеют различные энергии. Те электроны, энергия которых удовлетворяет условию $W < e\cdot V$, не могут достичь анода. Поэтому при увеличении $ |V|$ фототок уменьшается. При некотором значении $V = V_{3}$ (потенциал запирания) даже наиболее быстрые электроны не смогут достичь анода, и ток прекратится. При этом
\begin{equation}
	\label{eq:2}
	W_{max}=e\cdot V_{3}
\end{equation}
Из уравнений (\ref{eq:1}) и (\ref{eq:2}) найдём линейное соотношение между потенциалом запирания и частотой падающего света
\begin{equation}
	\label{eq:3}
	V_{3}=\frac he \nu-\phi
\end{equation}

Экспериментальная проверка формулы Эйнштейна была впервые осуществлена Ричардсоном и Комптоном в 1912 г., более тщательно - Милликеном в 1916 г. Обе работы подтвердили формулу (\ref{eq:3}). Наиболее точная проверка была проведена Лукирским и Прилежаевым в 1926г.

\section{Практическая часть}
\subsection{Описание эксперимента}
\begin{figure}[h!]
	\centering
	\includegraphics[width=\linewidth]{fig/fig2.png}
	\caption{Схема экспериментальной установки}
	\label{fig:2}
\end{figure}

Целью работы является проверка законов фотоэффекту а также измерение фундаментальной константы - постоянной Планка - на основании соотношения (\ref{eq:3}) между потенциалом запирания $V_{3}$ и частотой света $\nu$.

Схема установки приведена на рис. \ref{fig:2}. Свет от лампы накаливания или ртутной лампы фокусируется конденсором на входную щель призменного монохроматора УМ-2. В монохроматоре свет через входную щель, регулирующую световой потом, падает на объектив коллиматора и параллельным пучком проходит диспергирующую призму. Ввиду того, что фокусное расстояние объектива для каждой длины волны изменяется, предусмотрена возможность фокусировки объектива коллиматора. Фокусировочное движение осуществляется маховичком и контролируется по миллиметровой шкале с нониусом. В трубе коллиматора между щелью и объективом помещен затвор, с помощью которого можно прекратить доступ света в прибор. Под углом $90^{\circ}$ к падающему пучку света располагается выходная труба монохроматора. Поворачивая призменный столик на различные углы относительно падающего пучка света, получают свет различной длины волны, распространяющийся параллельно оси выходной трубы. Фокусируя свет в выходную плоскость, выделяют с помощью выходной щели узкий спектральный интервал. Прямо на выходной щели крепится фотоэлемент типа Ф-5, помещенный в металлический экран, предохраняющий от попадания паразитного света и от электрических наводок. Таким образом, поворачивая призму монохроматора, можно изменять частоту освещающего фотоэлемент света. На измерительном барабане поворотного механизма нанесены относительные деления-градусы поворота барабана. Отсчет читается против индекса, скользящего по спиральной канавке. Частота (длина волны) определяется по отсчету барабана монохроматора и градуировочному графику. Для градуировки прибора служит ртутная лампа.

Потенциал анода фотоэлемента можно менять с помощью двух потенциометров "Вел. V" и "Вел. $\Delta V$". Величина потенциала измеряется вольтметром при положении переключателя "Измерение" на "V" (шкала вольтметра при этом соответствует 2.5 В). В положении переключателя "$\Delta V$" измеряется лишь часть анодного напряжения, подаваемая с потенциометра "Вел. $\Delta V$" (шкала вольтметра на 0.25 В). Имеется переключатель знака анодного напряжения.

Возникающий в фотоэлементе ток при отрицательном потенциале анода (режим задержки) очень мал (порядка $10^{-11}$ А) и не может быть измерен непосредственно. Для его измерения служит балансный электрометрический усилитель. На вход усилителя подается напряжение, пропорциональное величине фототока, которое образуется на большом (порядка $10^9$ Ом) сопротивлении $R$, включенном в цепь фотоэлемента. Поэтому показания микроамперметра на выходе усилителя пропорциональны величине фототока.

Потенциал запирания определяется величиной тормозящего электроны анодного напряжения в момент исчезновения фототока. Следует подчеркнуть, что измеренное $V_{3}$ отличается от истинного значения
\begin{equation}
	V_0=V_{3}+C
\end{equation}
Величина поправки C зависит от ряда факторов, важнейшим из которых являются наличие обратного тока, вызванного вторичной электронной эмиссией с анода, и контактная разность потенциалов между анодом и катодом. Это приводит к большой погрешности 
при определении постоянной Планка по зависимости $V_{3}$ от $\nu$.

С целью увеличения точности опыта можно провести измерения для двух близких значений частоты света $\nu_1$ и $\nu_2$. Если частоты достаточно близки, то поправка $C$ практически не изменится (все остальные условия опыта останутся без изменения). Тогда разность $\Delta V_{3} = V_{3}(\nu_2)- V_{3}(\nu_1)$ будет совпадать с разностью истинных потенциалов запирания и, согласно уравнению (\ref{eq:3}), для определения постоянной Планка можно воспользоваться соотношением
\begin{equation}
	\Delta V_{3}=\frac he \Delta \nu,
\end{equation}
где $\Delta \nu= \nu_2- \nu_1$.

В этом методе необходимо достаточно точно знать значения частот $\nu_1$ и $\nu_2$. Точность определения частот по градуировочному графику монохроматора в случае применения источника белого света (лампа накаливания) невелика, поэтому необходимо использовать источник с линейчатым спектром, для которого частоты излучения измерены с высокой точностью. В данной работе применена ртутная лампа ДРШ-250. 

\subsection{Эксперимент}
\begin{figure}[H]
	\centering
	\includegraphics[scale=1]{scripts/Gradgraph.pdf}
	\caption{}
	\label{fig:3}
\end{figure}

\subsection{Задание 1}
Для трех различных значений длин волн снята зависимость фототока в режиме насыщения от интенсивности света с помощью изменения ширины входной щели монохроматора. 

Для щели 0,1 мм: $\nu=7,50*10^{14}$ ($\lambda=4000\buildrel _{\circ} \over {\mathrm{A}}$).

Для щели 0,06 мм: $\nu=6,55*10^{14}$ ($\lambda=4580\buildrel _{\circ} \over {\mathrm{A}}$).

Для щели 0,035 мм: $\nu=5.88*10^{14}$ ($\lambda=5102\buildrel _{\circ} \over {\mathrm{A}}$).

$V_3= 2.3 B.$

\begin{figure}[H]
    \centering
    \includegraphics[width=0.8\linewidth]{scripts/z11} 
    \caption{}
    \label{fig:4}
\end{figure}


Согласно закону Столетова, величина фототока в режиме насыщения при неизменном спектральном составе излучения прямо пропорциональна интенсивности падающего света. На построенном по результатам измерения графике получилась прямая, что подтверждает точность закона.

\subsection{Задание 2}
Снята зависимость тока фотоэлемента от частоты света, используя источник белого света.

$V_3= 2.3 B.$
\begin{figure}[H]
	\centering
	\includegraphics[scale=1]{scripts/z2.pdf}
	\caption{}
	\label{fig:5}
\end{figure}

Для каждого вещества существует длинноволновая ("красная") граница фотоэффекта $\nu_0$, за которой ($\nu < \nu_0)$ фотоэмиссия не наблюдается. Из результатов эксперимента: $\nu_0=2.37\pm 0.3*10^{14} \text{ Гц}$.

Для фотоэффекта, называемого {\itshape{нормальным}}, величина фототока монотонно убывает с длиной волны, стремясь к нулю на длинноволновой границе. Но если электрический вектор падающей волны {\itshape{не перпендикулярен}} к плоскости падения, то у ряда металлов на кривой зависимости фототока от длины волны наблюдается резко выраженный максимум в определенной области спектра. Фотоэффект, характеризующийся подобными максимумами на кривой называется {\itshape{селективным}}. 

Максимум на кривой напоминает резонансный максимум при вынужденных колебаниях гармонического осциллятора. Поэтому можно сказать, что электроны в металле при селективном эффекте ведут себя так, как если бы они обладали собственными частотами.

\subsection{Задание 3}
Построена зависимость $V_3$ от частоты света (выход 0.25 мм, вход 0.25 мм):
\begin{figure}[H]
	\centering
	\includegraphics[scale=1]{scripts/z3.pdf}
	\caption{}
	\label{fig:13}
\end{figure}
Согласно теории, соотношение между потенциалом запирания и частотой падающего света выглядит следующим образом: $$V_{3}=\frac he \nu-\phi$$

Отсюда находится постоянная планка: $h=6.032\pm 1.542*10^{-34}$ Дж$\cdot$с.

Красная граница фотоэффекта: $\nu_0=3.8\pm 0.3*10^{14} \text{ Гц}$.

\subsection{Задание 3}
Для всех значений $\Delta V_3$ вычислена постоянная Планка из соотношения $\Delta V_{3}=\frac he \Delta \nu$. В качестве источника света использовалась ртутная лампа.

Для щелей 0.3 мм

\begin{tabular}{ | l | l | l |}
\hline
$\Delta V_3, B$ & h, Дж$\cdot$с & $\Delta h$, Дж$\cdot$с \\ \hline
0.135 & $4.326*10^{-34}$ & $2.676*10^{-34}$ \\
0.23 & $12.283*10^{-34}$ & $3.819*10^{-34}$ \\
\hline
\end{tabular}

Для щелей 0.6 мм

\begin{tabular}{ | l | l | l |}
\hline
$\Delta V_3, B$ & h, Дж$\cdot$с & $\Delta h$, Дж$\cdot$с \\ \hline
0.245 & $7.851*10^{-34}$ & $4.790*10^{-34}$ \\
0.17 & $9.079*10^{-34}$ & $2.857*10^{-34}$ \\
\hline
\end{tabular}
 
Для щелей 0.9 мм

\begin{tabular}{ | l | l | l |}
\hline
$\Delta V_3, B$ & h, Дж$\cdot$с & $\Delta h$, Дж$\cdot$с \\ \hline
0.195 & $6.248*10^{-34}$ & $3.829*10^{-34}$ \\
0.133 & $7.076*10^{-34}$ & $2.256*10^{-34}$ \\
\hline
\end{tabular}

\section{Вывод}
В ходе эксперимента был проверен закон Столетова, согласно которому величина фототока в режиме насыщения при неизменном спектральном составе излучения прямо пропорциональна интенсивности падающего света. Также двумя способами была измерена красная граница для фотоэлемента типа Ф-5. Также проводились измерения постоянной Планка.

В первом способе красная граница находилась из графика зависимости тока фотоэлемента от частоты света: $\nu_0=3.8\pm 0.3*10^{14} \text{ Гц}$. 

Во втором способе красная граница находилась из графика зависимости запирающего напряжения от частоты света:  $\nu_0=2.37\pm 0.3*10^{14} \text{ Гц}$.

Согласно соотношению $\Delta V_{3}=\frac he \Delta \nu$ и из графика зависимости запирающего напряжения от частоты света: $h=6.032\pm 1.542*10^{-34}$ Дж$\cdot$с.

Из эксперимента с ртутной лампой (см.таблицы).


\end{document}
